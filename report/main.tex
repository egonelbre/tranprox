% document type
\documentclass [12pt]{article} % oneside

% fixes
\usepackage{fixltx2e} % LaTeX patches, \textsubscript
\usepackage{cmap} % fix search and cut-and-paste in PDF

% language
\usepackage[english]{babel}
\usepackage[T1]{fontenc}
\usepackage[utf8]{inputenc}

% usability
\usepackage {float}

% formatting
\usepackage{amsmath}
\usepackage{amsfonts}
\usepackage{url}
\usepackage{setspace}
\usepackage{hyperref}

\usepackage{fancyvrb}
\usepackage{mdwlist}
\usepackage{float,caption}

\usepackage{graphicx}
\usepackage{ifpdf}
\usepackage{listings, textcomp, color, xcolor, caption}

\title{Approximating discontinous functions}

\author{
    Egon Elbre \\
        Department of Computer Science \\
    University of Tartu\\
    Tartu\\
}

\date{\today}

\begin{document}

\maketitle

\begin{abstract}
Calculating discontinous functions can be difficult and computationally
expensive. By combining using approximation functions we can lower 
the computational expensiveness. Defining good approximation function 
can be a hard to get right. However, defining an approximation 
that works for some value can be simpler. We show how combining
several simple approximations can give us a better approximation
and reduce computational requirements.
\end{abstract}

\section{Introduction}

\paragraph{Outline}

\section{Theory}

\newcommand{\Real}{\mathbb{R}}
\newcommand{\defas}{ := }
\newcommand{\err}[1]{\varepsilon_{#1}}

We are trying to approximate a function $f$:

$$f : X \mapsto \Real$$

Although we explain the idea in $\Real$ set, the results also apply to 
any set that has transitive relation $<$ and $+-$ operation (TODO: find the correct algebraic structure). 
Using $<$ we can define $\min$ and $\max$ for that set.

$$ \min(x,y) \defas \begin{cases}
    x & \text{if $x \leq y$}, \\
    y & \text{otherwise}.
\end{cases}
$$

$$
\max(x,y) \defas \begin{cases}
    y & \text{if $x \leq y$}, \\
    x & \text{otherwise}.
\end{cases}
$$

X can be any set.

\subsection{Exact approximation}

Let's assume we are interested in range $R \subseteq X$. Let's assume we 
have functions $\alpha$ and $\beta$ such that

\begin{align*}
    \alpha(x) &\leq f(x), \forall x \in R \\
    \beta(x)  &\leq f(x), \forall x \in R 
\end{align*}

Now we can define $\epsilon$ function.

\begin{align*}
    \alpha(x) &= f(x) + \err\alpha, \err\alpha \geq 0 \\
    \beta(x)  &= f(x) + \err\beta, \err\alpha \geq 0
\end{align*}

It is trivial to derive function $\gamma$ where $\err\gamma$ is smaller 
than $\err\alpha$ and $\err\beta$.

\begin{align*}
    \gamma(x)   &= max(\alpha(x), \beta(x)) \\
                &= max(f(x) - \err\alpha(x), f(x) - \err\beta(x)) \\
                &= f(x) - min(\err\alpha(x), \err\beta(x)).
\end{align*}

This also gives us a usefulness requirement for $\alpha$ and $\beta$:

\begin{align*}
    \exists x \in R, ~ &\alpha(x) < \beta(x) \\
    \exists x \in R, ~ &\beta(x) < \alpha(x)
\end{align*}

This means that function $\alpha$ and $\beta$ must be complementary.
For some inputs one should give better approximations than the other.

\subsection{Probablisitic approximation}

\section{Approximating an unknown function}

\section{Examples}

Although the theory is straightforward the complexity arises from finding
the appropriate functions to combine.

\paragraph{Levenshtein Distance}

\paragraph{Clustering}

\section{Results}

\section{Conclusions}


\nocite{*}
\bibliographystyle{ieeetr}
\bibliography{bibliography}

\end{document}